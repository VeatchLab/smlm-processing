\documentclass{article}

\usepackage{amsmath,amssymb}

\usepackage{graphicx}

\title{Documentation for STORManalysis}
\author{Thomas Shaw}

\begin{document}
\maketitle

\section{Data files}
I submit, preliminarily, that data files should contain the following information:
\begin{itemize}
\item \verb|data| -- cell array with a standard dataset for each channel
\item \verb|units| -- units for \verb|x| and \verb|y| (e.g. pixels or nm)
\item \verb|source| -- what software produced this
\item \verb|date| -- date this was produced (by source)
\item \verb|filenames|
\end{itemize}

\section{Drift Correction}
Stage drift over the course of a localization-microscopy experiment can be
substantial, and correcting is important to obtaining optimal measurement
accuracy.

\subsection{Algorithms}

All of the algorithms available in this package rely on the assumption that
emitters and/or larger structures are fixed in place, but are detected many
times over the course of an experiment. Thus, a spatial cross-correlation of
localizations from two different time intervals should have a peak centered
at a displacement of 0. Normally, there will be a primary peak due to multiple
localizations of the same emitter (with width given by the localization precision).
Secondary peaks, also centered at 0, may also be present, due to larger structures
such as whole cells. The width of these peaks will depend on the length-scales of
the structures.

If there has been drift between the two time-intervals, the cross-correlation peak
will no longer be at 0 displacement, but will correspond to the drift.

\subsection{parameters to specify}
parameters are passed in a struct, with (at least) the following fields. All
distances should be specified in terms of the same units as are used for
\verb|x| and \verb|y| in the dataset.

\begin{itemize}
\item \verb|psize_for_alignment| -- Pixel size to be used for image correlations
\item \verb|npoints_for_alignment| -- How many time bins to use
\item \verb|nframes_per_alignment| -- How many frames go in each time bin
\item \verb|interp_method| -- METHOD option to pass to interp1 when interpolating between
    the measured shifts
\item \verb|rmax_shift| -- maximum allowable drift (in practice, this sets the
    size of the cross-correlation that is used for the first fit)
\item \verb|rmax| -- frame size to use for second fit (should be smaller than
    \verb|rmax_shift|)
\item \verb|update_reference| -- true or false. Add (shifted) data from each
    time bin to the reference to be used for subsequent shifts
\item \verb|include_diagnostics| -- Save more of the fitting data for diagnostics
\end{itemize}

\end{document}
